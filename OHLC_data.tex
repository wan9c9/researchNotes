\section{OHLC data}
\textit{2010.08.12}
Using OHLC data to analyze the dynamics of the stock returns.
(Lildholdt, 2002) considered a GARCH model.

Consider a stochastic process $ X $, let $ X_i(t) $ be its value at time $ i+f_i+t(1-f_i) $, where $ f_i\in [0,1] $ and $ t\in[0,1] $. Assume that
\begin{itemize}
\item  from $ X(i) $ to $ X(i+f_i) $, $ X $ unobservable except its values at the endpoints,
\item from $ X(i+f_i) $ to $ X(i+1) $, that is, $ X_i(t) $ for $ t\in[0,1] $ follow some general diffusions,
\[ dX_i(t)=\mu_i dt + \nu_i d W_t, \]
\[ d\sigma_t \]a Brownian motion with drift $ \mu_i $ and variance $ \sigma_i^2 $
\end{itemize}
\begin{align*}
r_i=&P_i(1)-P_i(0),\\
dP_i(t)=&\mu dt+\sigma_i dW(t), \ 0\leq t\leq 1,\\
\sigma_i^2=&\omega+\sum_j\alpha_j(r_{i-j}-\mu)^2+\sum_j \beta_j\sigma^2_{i-j},\\
a_i=&\sup_{t\in[0,1]}P_i(t)-P_i(0)\\
b_i=&P_i(0)-\inf_{t\in[0,1]}P_i(t)\\
\end{align*}
In this setting, the variance is determined by past squared closed value(the realized
There are other ways to define the volatility process. For example, (Chou, 2006) considered a separate regression of the daily upper range and lower range. 

We wish to establish a model which could handle with OPN, UPR, DNR, CLS, and VOL data. If we assume the price movement during a day is governed by a geometric Brownian motion. Note that this assumption is very common in the 
continuous-time stochastic process, but it is problematic to assume that the long-run price is a geometric Brownian motion, due to the change of information. However, it may be more plausible to assume that the price follows a GBM in a short time range, in which no apparent change of information occurs.

$( UPR_i, DNR_i, CLS_i ) $ can be characterized by the $ (\mu_i, \sigma_i) $,  we need only to specify the dymamics of $$ (OPN_i,\mu_i,\sigma_i, VOL_i)=(o_i,\mu_i,\sigma_i,v_i)\leftarrow ?(o_i,\mu_i,\log \sigma_i,v_i) $$

We would expect that the information before the trading of a day can be obsorbed in the movement of the the price in the day with some moises, which can have further influence with the trading in the sequent day.

How to compare the performance of two models?

Firstly, suppose the process is 



Volume should have positive impact on the volatility.
realized volatility
