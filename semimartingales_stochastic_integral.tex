
\section{local martingale}
\subsection{intro.}
A local martingale is a type of stochastic process satisfying the localized version of the martingale property.
\subsection{defi.}
Let $(\Omega,F,\mathcal{F},P)$ be a filtered probability space, let $X:[0,\infty)\times \Omega)\to S$ be a $\mathcal{F}$-adapted s.p.. Then $X$ is called an $\mathcal{F}$-local martingale if there exists a sequence of $\mathcal{F}$-stpping times $\tau_k:\Omega\to[0,\infty)$ s.t.
\begin{itemize}
  \item $P(\tau_k \uparrow)=1$, $\tau_k$ is a.s. strictly increasing;
  \item $P(\lim \tau_k=\infty)=1$;
  \item the stopped process $1_{\tau_k>0}X_t^{\tau_k}=1_{\tau_k>0}X_{t\wedge\tau_k}$ is a margingale for every $k$.
\end{itemize}
\subsection{e.g.}
Let $W_t$ be the standard Wiener process and $T=\inf\{t:W_t=-1\}$, then $ W_t^T=W_{t\wedge T} $ is a martingale,

\section{semimartingales}

$ X $ is a local martingale, if it is adapted, cadlag, and there exists a sequence of increasing stopping times, $ T_n $ s.t. $ \lim T_n=\infty, a.s. $, and $ \forall n $, $ X^{T_n} 1\{T_n>0 \} $ is uniformly integrable martingale.
Such $ T_n $ is called a fundamental sequence.

Topology generated by convergence of sequences.

An operator $ I_X $, to be an integral, should be linear and s.t. some version of bounded convergence theorem.

X is a total semimartingale, if X is cadlag, adapted, $ I_X $: $ S_u \to L^0 $ is continuous, where $ S_u $ is the set of simple predictable processes, and $ L^0 $ be the space of finite valued r.v. topologized by convergence in probability.

X is a semimartingale, if $ \forall t\in[0,\infty) $, $ X^t $, i.e., X stopped by t, is a total semimartingale.

The set of (total) semimartingales is a vector space.

Note that $ L^0 $ is dependent of the probability measure defined on it. If Q is a probability and is absolutely continuous w.r.t. P, and X is a P (total) semimartingale, then X is also a Q (total) semimartingale. (This is because Convergence in P-probability implies convergence in Q-probability, as $ P(An)\to 0 \rightarrow P(fI\{An\})\to 0) $, $ \forall f  $ integrable w.r.t. P.)

If X is a $ P_k $ semimartingale  for each k, and let $ P=\sum \lambda_k P_k $ with $ \lambda_k>=0 $ and $ \sum \lambda_k=1 $. Then X is a P semimartingale as well.

(Stricker's Theorem). Let X be a semimartingale for the filtration $ \mathbb{F} $, $ \mathbb{G} $ be a subfiltration of $ \mathbb{F} $ and  X is adapted to $ \mathbb{G} $. Then X is also a $ \mathbb{G} $ semimartingale. {\em Note for $ \mathbb{G} $ its test simple predictable processes is a subset of that of $ \mathbb{F} $.} Then we can always thrink a filtration and preserve the property of being a semimartingale as long as it is still adapted. Expanding the filtration, is a much delicate issue.  

(Jacod's countable expansion) Let $ \mathcal{A} $ be a collection of disjoint events in $ \mathcal{F} $. Let $ \mathcal{H}_t=\sigma\{\mathcal{F}_t,\mathcal{A}\} $. Then every $ (\mathbb{F}, P) $ semimartingale is an $ (\mathbb{H},P) $ semimartingale.  Particularly, this is true when $\mathcal{A} $ is a finite collections of events in $ \mathcal{F} $.

Note if B is a Brownian motion for a filtration, then we are able to add, in a certain manner, an infinite number of future events to the filtration and B will no longer be a martingale but it will stay a semimartingale. This has interesting implications in theory of continuous trading (Duffie-Huang).

Being a semimartingale is a local property. A process X is stoped at $ T- $ if $ X^{T-}=X_t 1\{0\leq t <T\}+X_{T-} 1\{t\geq T\} $, where $ X_{0-}=0 $. If X is a cadlag, adapted process. Let $ (T_n) $ be a sequence of positive r.v. increasing to $ \infty $ a.s., and $ (X^n) $ be a sequence of semimartingales s.t. $ X^{T_n-}=(X^n)^{T_n-} $, then X is a semimartingale.{\small \em Note $ T_n $ may not be stopping times. We need to show $ X^t $ is a total semimartingale for every $ t>0 $. Define $ R_n= T_n 1\{T_n\leq t\} +\infty 1\{T_n > t\} $, then $ P(|I_{X^t}(H)| \geq c)\leq P(|I_{(X^n)^t}(H)|\geq c) +P(R_n<\infty)$. } A corollary is that a process $ X $ s.t. there exists a sequence of stopping times $ (T_n) $ such that $ X^{T_n} $ or $ X^{T_n} 1\{T_n>0\} $ is a semimartingale for each $ n $, then $ X $ is a semimartingale.

Examples of semimartingales: adapted process with cadlag paths of finite variation on compacts (by definition), every $ L^2 $ martingale with cadlag paths. Cadlag, locally square integrable local martingale

