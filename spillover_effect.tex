\section{spillover effect}
Spillover effects are externalities of economic activity or processes upon those who are not directly involved in it. E.g.: the economic benefit of increased trade are the spillover effects anticipated in the formation of multilateral alliances of many of the regional nation states.

Analysis of comovement of stock trade may lead to prediction of stock price movement. 

Jorgensen et al: model with a single dynamic factor and no idio syncratic components.

Wedel, Bockenholt and Wagner (2003): a static multivariate Poisson factor model for cross-sectional\footnote{Cross-sectional data or cross section (of a study population) in statistics and econometrics is a type of one-dimensional data set. Cross-sectional data refers to data collected by observing many subjects (such as individuals, firms or countries/regions) at the same point of time, or without regard to differences in time. Analysis of cross-sectional data usually consists of comparing the differences among the subjects.} analyses.


